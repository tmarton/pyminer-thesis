\chapter{Klientska časť}
Táto časť poskytuje užívateľské rozhranie, naviguje užívateľa v rámci aplikácie, reaguje na jeho akcie a komunikuje s vystaveným rozhraním serverovej časti aplikácie.
\par Klienstka časť je naimplementovaná použitím frameworku Angular 2. Nejedná sa o tzv. single page ako je to bežné u aplikácií napísaných v tomto frameworku, ale je rozdelená na tri stránky, kde si každá drží svoj interný stav. Sú to stranký Miner, Mined Patterns a Regex Groups. Stránka Miner sa slúži na samotné odhaľovanie parsovacích vzorov, zatiaľ čo Stránky Mined Patterns a Regex Groups slúžia na prácu s finalizovanými parsovacími vzormi resp. sadami typov regulárnych výrazov, určených pre transformáciu vzorov.


\section{Angular 2}
Angular 2 je relatívne nový komponentovo-orientovaný JavaScriptový framework, ktorý odstraňuje nedostatky pôvodnej verzie. Aplikácie sa v Angular 2 píšu buď v Typescripte alebo vo verziách ECMAScript 5 resp. 6. Ani jeden variant zatiaľ nie je plne podporovaný dnešnými prehliadačmi, preto je nutná transkompilácia. My sme použili Typescript, nadstavbu JavaScriptu, ktorý zavádza statické typovanie a prvky objektovo-orientovaného programovania ako typy, triedy, moduly a pod., čo považujeme za veľkú výhodu. Základné stavebné prvky sú:

\begin{enumerate}
 \item Modul -- zpuzdruje funkcionalitu, ktorá vykonáva jednu úlohu. Funguje podobne ako OSGi modul, kde modul vidí len komponenty definované v samotnom module a komponenty, ktoré modul explicitne importuje. Zároveň modul špecifikuje, ktoré komponenty budú viditeľné zvonku balíčka.
 \item Komponent -- základný prvok aplikácie. Definuje aplikačnú logiku a interaguje so šablónou cez rozhranie vlastností a metód. Potrebná komfigurácia komponenty je zabezpečená cez metadáta.
 \item Šablóna -- html stránka obohatená o značky z jazyka Angular. Definuje, ako bude zobrazený komponent. Zároveň obsahuje značkovanie pre prepojenie dát, ktoré určuje, ako sú vymieňané dáta medzi komponentom a šablónou. V predošlej verzii Angularu sa vždy jednalo o obojsmerné prepojenie, čo spôsobovalo výkonnostné problémy. Vo verzii 2 si vieme toto prepojenie zašpecifikovať a vyhnúť sa tak týmto problémom.
 \item Servisná služba -- trieda, ktorá obsahuje funkcie na vykonanie špecifických úloh potrebných aplikáciou. 
 \item Dependency injection -- spôsob založený na rovnomennom návrhovom vzore, ktorým Angular poskytuje servisným službám a komponentom závislosti.
\end{enumerate}

\subsection{Podporné nástroje}
Aplikácie písané v Angular 2 sú závislé na knižniciach tretích strán ako aj na knižniciach samotného Angularu. 
\par Na správu týchto závislostí používame všeobecne polulárny Node.js a jeho balíčkovací systém npm, zatiaľ čo na správu samotného kódu používame Webpack, jednoduchý balíčkovač zdrojových kódov. Webpack ponúka široké možnosti, ako zdrojové kódy upraviť pred výsledným zabalením do archívu a zároveň možnosť vystaviť tieto kódy v rámci vývojového servera, čo sme pri vývoji vo veľkej miere využívali. V našom systéme Webpack pred výsledným zabalením zdrojové kódy minimalizuje a prevedie optimalizácie.
\par Oba tieto nástroje sme ovládali a nastavovali cez nástroj Angular CLI, ktorý umožnuje vygenerovať si projekt už obsahujúci najčastejšiu konfiguráciu. 


\subsection{Miner}
Na začiatku je možné určiť aplikáciu, z ktorej analyzované správy pochádzajú, a jej verziu. Ďalej môžeme určiť, či chceme analyzovať správy z logovacieho súboru, ktorý nahrajeme, alebo použijeme správy, ktoré už sú v systéme nahrané a neboli zatiaľ finálne analyzované.

\begin{figure}[htbp]
 \centering 
 \begin{minipage}{0.95\linewidth}
 	\centering
 	\missingfigure[figheight=5cm]{Menu aplikacie}
 	%\includegraphics[width=\textwidth]{images/RURC.png} 	
 \end{minipage}
  \caption{Trvanie eng.py }
  \label{fig:eng-duration}
\end{figure}

Po tomto kroku máme pred sebou rozhranie, ktorým budeme upravovať a spúšťať analýzu. 
\par Po upresnení oddeľovačov a q-percentilu užívateľ spustí analýzu. Po obdržaní výsledkov je užívateľovi prezentovaná tabuľka, ktorej hlavnú časť tvoria navrhované parsovacie vzory. V tabuľke napravo je tlačidlo, ktorým užívateľ daný parsovací vzor potvrdí ako finálny. Pre potreby spresňujúcej analýzy je pri každom vzore vľavo checkbox. Pri označení ľubovoľného počtu navrhovaných vzorov a následnej analýze sú ako vstupné správy brané len tie, ktoré prislúchali k označeným vzorom. Výsledky spresňujúcej analýzy sú vložené do tabuľky tak, že novo navrhované parsovacie vzory sú pripojené ako potomkovia vzoru, z ktorého vznikli, a vytvárajú tak stromovú štruktúru. Tento proces vieme opakovať, až kým nefinalizujeme všetky parsovacie vzory, alebo sa môžeme rozhodnúť pre začatie novej analýzy.

\begin{figure}[htbp]
 \centering 
 \begin{minipage}{0.95\linewidth}
 	\centering
 	\missingfigure[figheight=5cm]{Zadavanie source}
 	%\includegraphics[width=\textwidth]{images/RURC.png} 	
 \end{minipage}
  \caption{Trvanie eng.py }
  \label{fig:eng-duration}
\end{figure}

\begin{figure}[htbp]
 \centering 
 \begin{minipage}{0.95\linewidth}
 	\centering
 	\missingfigure[figheight=5cm]{Proces analyzy}
 	%\includegraphics[width=\textwidth]{images/RURC.png} 	
 \end{minipage}
  \caption{Trvanie eng.py }
  \label{fig:eng-duration}
\end{figure}

\subsection{Mined patterns}
Rozhranie prezentuje stránkovací zoznam finalizovaných parsovacích vzorov, ktoré si je možné vyfiltrovať podľa zdrojovej aplikácie a jej verzie. Aktuálny vyfiltrovaný zoznam si užívateľ vie exportovať do formátu REtrie. Zoznam ďalej umožnuje náhľad na správy, ktoré boli vygenerované použitím vybraného parsovacieho vzoru. 

\begin{figure}[htbp]
 \centering 
 \begin{minipage}{0.95\linewidth}
 	\centering
 	\missingfigure[figheight=5cm]{Obrazok patternov}
 	%\includegraphics[width=\textwidth]{images/RURC.png} 	
 \end{minipage}
  \caption{Trvanie eng.py }
  \label{fig:eng-duration}
\end{figure}

\subsection{Regex Groups}
Zobrazuje zoznam sád typov regulárnych výrazov, ktoré sú v aplikácií nahrané. V systéme sme prednastavili základnú sadu, ktorá by mala byť postačujúca pre značnú časť prípadov a táto sada nejde zo systému zmazať. Systém ďalej umožnuje nahrať novú sadu zo súboru vo formáte:

 \begin{figure}[htbp]
 \centering 
 \begin{minipage}{0.95\linewidth}
 	\centering
 	\missingfigure[figheight=7cm]{Regex group}
 	%\includegraphics[width=\textwidth]{images/RURC.png} 	
 \end{minipage}
  \caption{Trvanie eng.py }
  \label{fig:eng-duration}
\end{figure}
