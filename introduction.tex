\chapter{Úvod}

Už od vynálezu počitačov produkuje každý softwér záznam o svojom behu. Tieto záznamy sa volajú logové súbory a slúžia programátorom a systémovým administrátorom na odladenie chýb programu a jeho monitorovanie. Logové súbory sú vytvárané programátormi vo forme viet hovorového jazyka, obsahujúcich všetky dôležité informácie o udalosti v programe, ktorá práve nastala. Vo výsledku teda obsahujú jedinečné kvantum informácií, ktoré nie sú inak dostupné. Kvôli rozmachu technológií sa však rapídne zvýšil objem logovacích súborov, v ktorých už viac nie je možné triviálne nájsť požadované informácie. V praxi sa preto stretávame s prípadom, že síce máme data o udalostiach, ktoré nastali, ale nevieme ich nájsť a správne použiť, alebo ani nevieme, že tieto údaje máme. Sme teda bohatí na data, ale chudobní na informácie.
\par Z tohto dôvodu je nutné vedieť informácie o udalosti v programe z logových súborov efektívne a automaticky extrahovať a predpripraviť na ďalšie spracovanie. To sa deje vytvorením parsovacích vzorov, ku ktorým vieme každú udalosť jednoznačne priradiť. Momentálne však neexistuje nástroj, ktorý by takéto vzory vytvoril nad ľubovoľnou množinou vstupných logovacích súborov v dostatočnej kvalite. Existujúce algoritmy trpia na rôzne nedostatky, ako aj na potrebu detailného nastavenia vstupných parametrov.
\par V práci sa zaoberáme rozšírenou verziou algoritmu Nagappan-Vouk, vyvinutou vedúcim práce RNDr. Danielom Tovarňákom. Tento algoritmus slúži na odhaľovanie parsovacích vzorov v logovacích súboroch. Vychádza z algoritmov popísaných v práci, pričom odstraňuje ich známe nedostatky za pridania jednoduchej užívateľskej interakcie. Cieľom tejto práce je teda naimplementovať užívateľské rozhranie, ktoré bude plne podporovať algoritmus Nagappan-Vouk, pokúsiť sa zlepšiť presnosť výsledných parsovacích vzorov a porovnať výsledky voči podobným algoritmom.
 \par Práca sa člení ....