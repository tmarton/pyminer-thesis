\chapter*{Úvod}
\addcontentsline{toc}{chapter}{\textbf{Úvod}}

Už od vynálezu počitačov produkuje každý softwér záznam o~svojom behu. Tieto záznamy sa volajú logové subory a slúžia pre programátorov a systémových administrátorov na odladenie chýb programu a jeho monitorovanie. Logové súbory sú vytvárané programátormi vo forme viet hovorového jazyka obsahujúcich všetky dôležité informácie o~udalosti v~programe, ktorá práve nastala. Vo výsledku teda logovacie súbory obsahujú jedinečné kvantum informácií, ktoré nie sú inak dostupné. Kvôli rozmachu technológií sa však rapídne zvýšil objem logovacích súborov, v~ktorých už viac nie je možné triviálne nájsť požadované informácie. V~praxi sa preto stretávame s~prípadom, že síce máme data o~udalostiach, ktoré nastali, ale nevieme ich nájsť a správne použiť, alebo ani nevieme, že tieto údaje máme. Sme teda bohatí na data, ale chudobní na informácie.
\par Je nutné vedieť efektívne a automaticky tieto informácie z~týchto súborov extrahovať a predpripraviť na ďalšie spracovanie. Toto sa deje vytvorením parsovacích vzorov, ku ktorým vieme každú udalosť jednoznačne priradiť. Momentálne neexistuje nástroj, ktorý by takéto vzory vytvoril nad ľubovoľnou množinou vstupných logovacích súborov v~dostatočnej kvalite. Existujúce algoritmy trpia na rôzne nedostatky, ako aj na potrebu detailného nastavenia vstupných parametrov.
\par V~práci sa zaoberáme rozšírenou verziou algoritmu Nagappan-Vouk, vyvinutou vedúcim práce, ktorý slúži na odhaľovanie parsovacích vzorov v~logovacích súboroch. Tento algoritmus vychádza z~algoritmov popísaných v~práci, pričom odstraňuje ich  známe nedostatky za pridania jednoduchej užívateľskej interakcie. Cieľom práce je naimplementovať užívateľské rozhranie, ktoré bude plne podporovať tento algoritmus, pokúsiť sa zlepšiť presnosť výsledných parsovacích vzorov a porovnať výsledky voči podobným algoritmom.
 \par Práca sa člení na osem kapitol. V~prvej kapitole sa venujeme problematike všeobecného logovania. Druhá kapitola približuje algoritmy hľadania parsovacích vzorov. V~kapitolách tri a štyri sa práca venuje popisu algoritmu Nagappan-Vouk a jeho vlastností. V~ďalšej kapitole analyzujeme požiadavky na vyvýjaný systém. Kapitoly šesť až osem popisujú implementáciu navrhnutého riešenia. V~závere zhodnotíme ciele práce.