\chapter{Inštalácia a nasadenie}
Implementovaná aplikácia používa len voľne dostupné knižnice a nástroje, ktoré sú schopné bežat na všetkých majoritných operačných systémoch. Predpokladáme ale, že aplikácia bude hlavne nasadená na linuxový server, ako je to bežné. Preto sme vytvorili scripty, ktoré by mali administrátorovi pomôcť automatizovať proces nasadenia na linuxový server RHEL len za použitia nevyhnutne minimálnej konfigurácie. Použili sme pritom konzolové nástroje Fabric a Ansible. Oba tieto nástroje sú využívané na automatizované nasadenie cez ssh. Fabric je priamočíarejší a príkazy sú v ňom zadávané deklaratívne ako volania python funkcií. Ansible je síce deklaratívny, ale vyžaduje dlhší čas na pochopenie ako Fabric. Ponúka širšie možnosti a paralelnú prácu s viacerými vzdialenými strojmi prostredníctvom konfigurácie umiestnenej v yaml súboroch.  My sme sa rozhodli oba tieto nástroje skombinovať. Ansible použijeme pre inštaláciu a spustenie aplikácie na vzdialenom servri. Fabric zase ako wrapper, ktorý nainštaluje závislosti pre Ansible a spustí ho. Oba tieto procesy bežia bez užívateľskej interakcie a nastavenia berú z konfiguračného súboru, ktorý obsahuje tieto nastavenia:

\begin{enumerate}
  \item host - adresa vzdialeného serveru, na ktorom bude prevedené nasadanie
  \item root - účet na vzdialenom serveri s rootovskými právami
  \item root \textunderscore password - heslo užívateľa s rootovskými právami
  \item  user \textunderscore name - užívateľ, pod ktorým sa nasadí a pustí výsledná aplikácia
  \item  user \textunderscore password - heslo pre novovytvoreného užívateľa
  \item user \textunderscore group - skupina novovytvoreného užívateľa
\end{enumerate}


\section{Príprava prostredia}
V~tejto fáze potrebujeme mať prístupný interpret jazyka Python vo verzii 2 ako prerekvizitu Fabricu. Tento interpret bude použitý len na spustenie Fabric procesov, je teda rozlišný od verzie, nad ktorou pobeží aplikácia. Samotná príprava postupuje v nasledujúcich krokoch:

\begin{enumerate}
  \item Vygenerujeme ssh kľúče
  \item Vytvoríme užívateľa a skupinu, pod ktorou bude bežať naša aplikácia
  \item Priradíme vytvorenému užívateľovi vytvorené kľúče a reštartujeme ssh démona
  \item Nahrajeme vygenerované ssh kľúče na vzdialený server
  \item Nainštalujeme samotný Ansible
\end{enumerate}

Tento proces sa deje výlučne na lokálnom počítači a po jeho skončení máme vytvorený nový užívateľský účet, pomocou ktorého Ansible v ďalšich krokoch cez ssh nasadí a spustí vytvorenú aplikáciu. Postupnosť týchto krokov nie je idempotentná, a preto ju na rozdiel od samotného nasadenia je možné vykonať s rovnakými nastaveniami len raz bez chybového výstupu.

\section{Ansible nasadenie}
Ansible používa pre konfiguráciu súbor nazývaný playbook.  Ako môžeme vidieť na príklade nasadenia aplikácie Pyminer \ref{fig:playbook},  playbook má modulárnu štruktúru a skladá sa z podsekcií:

\begin{enumerate}
  \item hosts
  \item variables
  \item tasks(roles)
\end{enumerate}

\par Sekcia hosts obsahuje IP adresy alebo doménové mená všetkých vzdialených servrov, s ktorými playbook pracuje. Tieto adresy sa ešte môžu členiť na ďalšie skupiny.
\par Variables obsahuje dvojice typu kľúč-hodnota, ktoré sú použité v playbooku.
\par Samotná logika je umiestnená v časti tasks. Zakladná jednotka je task, ktorý vykoná jeden zadaný príkaz. Ansible má predpripravené príkazy pre širokú škálu operačných systémov, frameworkov a nástrojov pre čo najpohodlnejšiu konfiguráciu, a zároveň umožnuje aj volanie funkcií daného operačného systému. Pre tasky je prístupný aj komplexný šablonovací jazyk a systém handlerov, ktorý umožnuje reagovať na určité udalosti.  Rola je skupina taskov zoskupených dokopy tak, že logicky vytvárajú jeden. Napr. tasky pre inštaláciu, konfiguráciu a spustenie Nginx serveru môžeme zoskupiť do role nginx, ako je ukázané v ref{fig:role}. 

\begin{figure}[htbp]
\centering
\begin{minipage}{0.9\textwidth}
\lstset{columns=flexible,breaklines=true,breakatwhitespace=true, showstringspaces=false}
\begin{lstlisting}
- name: apply common configuration to server
  hosts: all
  user: deployer
  roles:
    - common
    - postgres
    - redis
    - server
    - client
    - nginx
 vars:
  	- app_name: pyminer
\end{lstlisting} 		
\end{minipage} 
\caption{Ukážka playbooku pre nasadenie aplikácie Pyminer}
\label{fig:playbook}
\end{figure}

\begin{figure}[htbp]
\centering
\begin{minipage}{0.9\textwidth}
\lstset{columns=flexible,breaklines=true,breakatwhitespace=true, showstringspaces=false}
\begin{lstlisting}
- name: Set up nginx config
  dnf: name=nginx state=latest
  become: yes

- name: Write nginx conf file
  template: src=nginx.conf dest=/etc/nginx/conf.d/{{ app_name }}.conf
  become: yes
  notify:
    - restart nginx
\end{lstlisting} 		
\end{minipage} 
\caption{Ukážka role pre inštaláciu Nginx}
\label{fig:role}
\end{figure}
 
Playbook pre nasadenie našej aplikácie postupuje nasledovne:

\begin{enumerate}
  \item Pomocou balíčkovacieho manažéra dnf nainštaluje všetky potrebné závislosti aplikácie na vzdialenom serveri
  \item Nahrá zdrojové súbory na vzdialený server
  \item Nainštaluje PostreSQL a Redis ako služby a spustí ich
  \item Vytvorí virtuálne prostredie pre python a pomocou pip nainštaluje potrebné python knižnice
  \item Nainštaluje a spustí Gunicorn - WSGI server pre python
  \item Stiahne si inštaláciu Node.js, nainštaluje ju a zkompiluje klientsku aplikáciu v~produkčnom nastavení
  \item Nainštaluje Nginx a nastaví ho tak, aby poskytoval statické súbory priamo, a zároveň slúžil ako reverse proxy pre Gunicorn server
\end{enumerate}


Tento script je idempotentný, a preto je možné vždy prenasadiť novú verziu aplikácie.
Týmto spôsobom sme plne zautomatizovali proces nasadenia aplikácie. 



